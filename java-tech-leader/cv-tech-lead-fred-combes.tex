%%%%%%%%%%%%%%%%%%%%%%%%%%%%%%%%%%%%%%%%%
% Frédéric Combes/CV
% XeLaTeX Template
% Version 1.0 (2014-12-06)
%
% This template has been downloaded from:
% http://www.LaTeXTemplates.com
%
% Original author:
% Adrien Friggeri (adrien@friggeri.net)
% https://github.com/afriggeri/CV
%
% License:
% CC BY-NC-SA 3.0 (http://creativecommons.org/licenses/by-nc-sa/3.0/)
%
% Important notes:
% This template needs to be compiled with XeLaTeX and the bibliography, if used,
% needs to be compiled with biber rather than bibtex.
%
%%%%%%%%%%%%%%%%%%%%%%%%%%%%%%%%%%%%%%%%%

\documentclass{friggeri-cv} 	% Add 'print' as an option into the square bracket to remove colors from this template for printing

\usepackage[french]{babel}
\frenchbsetup{StandardItemLabels}

\usepackage{csquotes}

\hypersetup{pdfborder={0 0 0},colorlinks}
\newcommand*{\myAddress}{
APT 111, BAT C\\
46 Chemin de l’église de Montaudran\\
31400 TOULOUSE\\
France
}

\newcommand*{\myPhoneNumber}{+33 (0)6 31 97 71 06}

\newcommand*{\myLanguages}{
français \textit{maternelle}\\
anglais \textit{(lu, écrit, parlé)}
}

\newcommand*{\myReferences}{
\href{mailto:marthym@gmail.com}{marthym@gmail.com}
\href{https://github.com/Marthym}{github.com/Marthym}
\href{https://www.linkedin.com/in/combesfrederic}{in/combesfrederic}
}

\begin{document}

\headernamefirst{frédéric}{|}{combes}{java technical leader}

%----------------------------------------------------------------------------------------
%	SIDEBAR SECTION
%----------------------------------------------------------------------------------------

\begin{aside} % In the aside, each new line forces a line break
\section{compétences}
Java 21, Spring Boot,
Maven, Git, Docker,
MySQL, Mongo, Postgres
Agile, Scrum,
Linux
\section{contact}
\myAddress
~
\myPhoneNumber
\myReferences
\section{langues}
\myLanguages
\end{aside}

%----------------------------------------------------------------------------------------
%	SECTION: 'Expérience Professionnelle'
%----------------------------------------------------------------------------------------
\section{expériences professionnelles}

\begin{entrylist}
%---- BOTdesign --------------------------------------------------------------------
\entry
{Depuis 2022}
{CTO Hands-On}
{Toulouse}
{\vspace{-0.2cm}\emph{BOTdesign} \markedright{/ Management, Organisation, NodeJS, Java, Spring, MongoDB, Ansible /}\\

En tant que CTO hands-on chez BOTdesign, j’ai eu l’occasion de \textbf{mener l’équipe technique} au travers de la modernisation et du développement 
d'une application dispositif médical.

J’ai eu l’occasion de travailler sur des projets divers comme :
\begin{enumerate}
    \item Automatisation du \textbf{déploiement et sécurisation des infrastructures} Azure
    \item Architecture et implémentation d’un \textbf{refactor de l’application} (NodeJS vers Java 21 / Spring)
    \item Implémentation et déploiement d’un système de BI (Metabase / Postgres)
    \item Mise en place de processus agile et structuration de l’équipe technique
\end{enumerate}

\
}

%---- I-RUN.FR --------------------------------------------------------------------
\entry
{2018 \ding{224} 2022}
{Technical Leader Backend}
{Toulouse, Launaguet}
{\vspace{-0.4cm}\emph{i-Run.fr} \markedright{/ Spring Boot, Java 11, Vue.js, Docker, Linux, MySQL, Agile /}\\
\begin{description}[leftmargin=0cm]
    \item [\hspace*{-1cm}\bodyfont{|} \normalfont \textbf{\color{orange}CTO \color{headercolor}Adjoint}] \hfill \textit{Avril 2020}\\
        Je participe aux développements \textbf{Java 11},valide \textbf{l’architecte logicielle} du projet de refonte et organise la mise en place 
        du monitoring des applications grâce à \textbf{Prometheus et Grafana}. Je mets en place un plan de migrations afin de moderniser et sécuriser
        les infrastructures matérielles. 

    \item [\hspace*{-1cm}\bodyfont{|} \normalfont \textbf{\color{orange}Responsable \color{headercolor}Technique}] \hfill \textit{Avril 2019}\\
        J’ai initié et porté le projet de refonte et ré-architecture logicielle des applications. J’ai formé les équipes 
        à \textbf{Java 11, Reactor et Spring Weblux} pour développer de nouveaux composants. 
        J’ai porté les concepts d’\textbf{architecture hexagonale} auprès des équipes de développement.

    \item[\hspace*{-1cm}\bodyfont{|} \normalfont \textbf{\color{orange}Tech \color{headercolor}Lead Backend}] \hfill \textit{Janvier 2018}\\
        Ma première tache fut de sécuriser le système de déploiement. J’ai \textbf{uniformisé et automatisé les builds} de tous les projets 
        et développé un CI dans gitlab qui permet aujourd’hui de déployer l'ensemble des applications en un click. En parallèle, 
        je participe aux développements et à la maintenance des applications \textbf{Java 8}. 
\end{description}
\
}

%---- LIVING OBJECTS --------------------------------------------------------------------
\entry
{2014 \ding{224} 2017}
{Expert développement Java}
{Toulouse, Basso Cambo}
{\vspace{-0.4cm}\emph{Living Objects} \markedright{/ Java 8, Neo4j, Maven, Docker, Scrum /}\\
\begin{enumerate}[leftmargin=0cm]
    \item Membre d’une équipe d’une \textbf{quinzaine de développeurs}, je travaille sur le produit principal de la société.
    \item Je participe à l’élaboration de l’\textbf{architecture micro-service REST} du produit.
    \item Je développe un composant logiciel en \textbf{Java 8}, sur un \textbf{framework OSGi} pour le serveur REST. 
        Le stockage utilise une \textbf{base de données Neo4j} pour représenter les connexions entre les routeurs du réseau mobile SFR.
\end{enumerate}
\
}
\end{entrylist}
\begin{entrylist}
%---- CAMELEON SOFTWARE -----------------------------------------------------------------
\entry
{2007 \ding{224} 2013}
{Consultant Technique Expert}
{Toulouse, Labège}
{\vspace{-0.4cm}\emph{Cameleon Software} \markedright{/ Java J2EE, Struts, Jenkins, VMWare, Oracle /}\\
\begin{enumerate}[leftmargin=-0cm]
    \item Expert dans des \textbf{équipes de 10 à 20 personnes}, je développe des \textbf{composants d’interface responsive en JSP} et 
        je forme les équipes à l’intégration de solution e-Commerce pour de grosses compagnies d’assurance comme MNT, MMA ou AXA.
    \item \textbf{Missions à l’étranger} pour travailler avec notre équipe à Chicago sur des projets SaS comme Clear Channel et Crystal.
\end{enumerate}
\
}
\end{entrylist}
\begin{entrylist}
%---- ACCESS COMMERCE --------------------------------------------------------------------
\entry
{2001 \ding{224} 2007}
{Développeur d'adaptations}
{Toulouse, Labège}
{\vspace{0.2cm}\emph{Access-Commerce}  \markedright{/ Java, J2EE, Administration Oracle, Linux/UNIX /}\\
En tant que ressource technique d’une équipe de sept personnes, j’ai, entre autres, effectué les
missions suivantes : Installation et administration de serveur Oracle, développement de portail de vente en ligne, 
Développements et adaptations diverses grâce aux technologies Oracle.
}
%------------------------------------------------
\end{entrylist}

%----------------------------------------------------------------------------------------
%	Section FORMATION
%----------------------------------------------------------------------------------------

\section{formation}

\begin{entrylist}
%------------------------------------------------
\entry
{2006}
{Certification Oracle {\normalfont 10g Associate}}
{Oracle, Paris}
{}
%------------------------------------------------
\entry
{2001}
{Diplôme Universitaire {\normalfont Nouvelles technologies}}
{IUT de Rodez}
{Année de spécialisation Internet et nouvelles technologies}
%------------------------------------------------
\entry
{2000}
{DUT {\normalfont Informatique Génie logiciel}}
{IUT de Rodez}
{}
%------------------------------------------------
\entry
{1998}
{Bac S {\normalfont Spécialité Biologie}}
{Lycée Jeanne d’Arc de Millau}
{}
%------------------------------------------------
\end{entrylist}

%----------------------------------------------------------------------------------------
%	Section COMPETENCES
%----------------------------------------------------------------------------------------

\section{compétences}
\vspace{-0.4cm}
\begin{capabilitize}
	\capability {Anglais}
		{Lu, écrit et parlé}
		{Utilisé lors de mes déplacements aux USA et durant les réunions de cadrage avec les clients
		étrangers.}
	\capability {Java 21 / Maven}
		{Expert}
		{Langage de programmation principal des applications éditées par Caméléon Software, Living Objects \& I-Run.fr}
	\capability {Spring / Spring boot}
		{Expert}
		{Framework principal des application chez I-Run.fr}
	\capability {Angular / Vue.js}
		{Connaissance}
		{Framework front utilisé chez i-Run.fr et BOTdesign}
	\capability {Agile, Scrum}
		{Maîtrise}
		{Méthodologie utilisée avec souplesse chez Living Objects et mis en place chez i-Run.fr et BOTdesign}
	\capability {Oracle, MySQL, MongoDB}
		{Expert}
		{Certification OCA 10g, MySQL comme base de données principale chez Living Objects et i-Run.fr, MongoDB comme base de données principale chez BOTdesign}
	\capability {Linux}
		{Expert}
		{Unique système d’exploitation utilisé chez Cameleon Software \& Living Objects}
\end{capabilitize}

%----------------------------------------------------------------------------------------
%	Section INTERETS
%----------------------------------------------------------------------------------------

\section{divers \& centres d’intérêt}
\vspace{-0.4cm}
\begin{enumerate}
    \item Détenteur d’un \textbf{permis B}
    \item \textbf{Professeur vacataire} à l’IUT Informatique de Toulouse
    \item \textbf{professionnel:} travail en équipe, challenges techniques, algorithmique, TDD
    \item \textbf{personnel:} Membre du bureau de l’association sportive de Volley de Montaudran, VLAM.
\end{enumerate}

\end{document}
