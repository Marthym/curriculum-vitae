%%%%%%%%%%%%%%%%%%%%%%%%%%%%%%%%%%%%%%%%%
% Frédéric Combes/CV
% XeLaTeX Template
% Version 1.0 (2014-12-06)
%
% This template has been downloaded from:
% http://www.LaTeXTemplates.com
%
% Original author:
% Adrien Friggeri (adrien@friggeri.net)
% https://github.com/afriggeri/CV
%
% License:
% CC BY-NC-SA 3.0 (http://creativecommons.org/licenses/by-nc-sa/3.0/)
%
% Important notes:
% This template needs to be compiled with XeLaTeX and the bibliography, if used,
% needs to be compiled with biber rather than bibtex.
%
%%%%%%%%%%%%%%%%%%%%%%%%%%%%%%%%%%%%%%%%%

\documentclass{friggeri-cv} 	% Add 'print' as an option into the square bracket to remove colors from this template for printing

\usepackage[french]{babel}
\frenchbsetup{StandardItemLabels}

\usepackage{csquotes}

\hypersetup{pdfborder={0 0 0},colorlinks}
\newcommand*{\myAddress}{
APT 111, BAT C\\
46 Chemin de l’église de Montaudran\\
31400 TOULOUSE\\
France
}

\newcommand*{\myPhoneNumber}{+33 (0)6 31 97 71 06}

\newcommand*{\myLanguages}{
français \textit{maternelle}\\
anglais \textit{(lu, écrit, parlé)}
}

\newcommand*{\myReferences}{
\href{mailto:marthym@gmail.com}{marthym@gmail.com}
\href{https://github.com/Marthym}{github.com/Marthym}
\href{https://www.linkedin.com/in/combesfrederic}{in/combesfrederic}
}

\begin{document}

\header{Chief Technical Officer}{frédéric}{|}{combes}

%----------------------------------------------------------------------------------------
%	SIDEBAR SECTION
%----------------------------------------------------------------------------------------

\begin{aside} % In the aside, each new line forces a line break
\section{contact}
\myAddress
~
\myPhoneNumber
~
\myReferences
\section{langues}
\myLanguages
\section{compétences}
Management, Leadership,
Communication,
Bienveillance,
Agile, Scrum,
Bagage Technique
\end{aside}

%----------------------------------------------------------------------------------------
%	SECTION: 'Expérience Professionnelle'
%----------------------------------------------------------------------------------------
\section{expériences professionnelles}

% Résumé à retoucher
CTO innovant avec plus de \textbf{5 ans d'expérience} dans la direction des équipes techniques et de la mise en œuvre de 
technologies et de solutions de pointe. Habile à \textbf{recruter et à gérer des équipes performantes}. 
Habitué à travailler dans des environnements à \textbf{forte contraintes de sécurité}.
Je cherche à mettre à profit mon expertise et mon expérience au service de nouveaux challenges.

\begin{entrylist}
%---- BOTdesign --------------------------------------------------------------------
\entry
{Depuis 2022}
{Chief Technical Officer}
{Toulouse}
{\vspace{-0.2cm}\emph{BOTdesign} \discretright{/ Management, Organisation, Agilité, Dispositif Médical, NodeJS, Java, Spring, Cloud /}\\

En tant que CTO chez BOTdesign, j’ai eu la charge de \textbf{restructurer l’équipe de développement} et de redonner au dispositif médical HEKO une nouvelle
dimension produit. Avec une équipe technique de 5 personnes, j'ai établi une \textbf{roadmap solide et réaliste} pour répondre aux besoins fonctionnels et techniques
de clients.

Ma capacité à \textbf{mener des équipes} et à donner une direction m’a permis de recruter une équipe bienveillante et soudée capable d’assumer des projets de
refonte importants aussi bien que la maintenance quotidienne de toute la plateforme.

En tant que membre de la direction, je participe et oriente les \textbf{décisions stratégiques} prises pendant le comité de direction de la société.
\\
}

%---- i-RUN.FR --------------------------------------------------------------------
\entry
{2018 \ding{224} 2022}
{CTO Adjoint}
{Toulouse, Launaguet}
{\vspace{-0.2cm}\emph{i-Run.fr} \discretright{/ Management, Organisation, Agilité, Background Technique /}\\

En 4 ans chez i-Run.fr j’ai eu l’opportunité d’évoluer sur trois niveaux d’encadrement d’équipes et ainsi d’affiner mes \textbf{compétences en management et mon leadership}.
\begin{description}[leftmargin=0cm]
    \item [\hspace*{-1cm}\bodyfont{|} \normalfont \textbf{\color{orange}CTO \color{headercolor}Adjoint}] \hfill \textit{Avril 2020}\\
    En tant qu’adjoint du DSI, je manage et coordonne \textbf{20 personnes dans 4 équipes}. De l’équipe fonctionnelle qui recueille le besoin,
    jusqu’aux SRE qui déploient et sécurisent les applications.
    \item [\hspace*{-1cm}\bodyfont{|} \normalfont \textbf{\color{orange}Responsable \color{headercolor}Technique}] \hfill \textit{Avril 2019}\\
    La société me fait confiance et me nomme au poste de Responsable Technique. Je manage alors l’ensemble des 8 développeurs. 
    Je \textbf{supervise l’ensemble des choix techniques}.
    \item[\hspace*{-1cm}\bodyfont{|} \normalfont \textbf{\color{orange}Tech \color{headercolor}Lead Backend}] \hfill \textit{Janvier 2018}\\
    En tant que Tech Lead Backend, je travaille avec mon homologue Frontend afin de proposer une stratégie et une architecture logicielle cohérente
    pour les futurs produits de la société.
\end{description}
\
}

%---- LIVING OBJECTS --------------------------------------------------------------------
\entry
{2014 \ding{224} 2017}
{Expert développement Java}
{Toulouse, Basso Cambo}
{\vspace{0.2cm}\emph{Living Objects} \discretright{/ Agile, Scrum, Java 8, Maven, Docker /}\\
Au sein du département Recherche \& Développement, je participe au développement du produit principal de la société.\\
L’autonomie laissée à notre équipe de 15 développeurs me permet de mettre à profit ma \textbf{capacité à prendre du recul}. Ma \textbf{vision d’ensemble} des projets m’aide à proposer des \textbf{choix techniques et des solutions appropriées} aux difficultés rencontrées.
\\}
\end{entrylist}
\begin{entrylist}
%---- CAMELEON SOFTWARE -----------------------------------------------------------------
\entry
{2007 \ding{224} 2013}
{Consultant Technique Expert}
{Toulouse, Labège}
{\vspace{0.2cm}\emph{Cameleon Software} \discretright{/ Java J2EE, Struts, Jenkins /}\\
Référent Technique au sein de l’équipe "CEO Services" composée de 9 personnes, je suis chargé de \textbf{comprendre et d’implémenter les besoins de nos clients} en intégrant la solution logicielle développée par l’entreprise. En contact direct avec les clients, les prestataires et les équipes de développement, j’ai la charge de \textbf{coordonner les ressources techniques} pour amener les projets à terme dans les délais requis.
\
}

%---- ACCESS COMMERCE --------------------------------------------------------------------
\entry
{2001 \ding{224} 2007}
{Développeur Java / Oracle}
{Toulouse, Labège}
{\vspace{0.2cm}\emph{Access-Commerce}  \discretright{/ Java, J2EE, Administration Oracle, Linux/UNIX /}\\
En tant qu’unique ressource technique d’une équipe de 7 personnes, j’effectue, entre autres, des développements et adaptations diverses sur l’ERP intégré par la société.
}
%------------------------------------------------
\end{entrylist}

%----------------------------------------------------------------------------------------
%	Section FORMATION
%----------------------------------------------------------------------------------------

\section{formation}

\begin{entrylist}
%------------------------------------------------
\entry
{2006}
{Certification Oracle {\normalfont 10g Associate}}
{Oracle, Paris}
{}
%------------------------------------------------
\entry
{2001}
{Diplôme Universitaire {\normalfont Nouvelles technologies}}
{IUT de Rodez}
{Année de spécialisation Internet et nouvelles technologies}
%------------------------------------------------
\entry
{2000}
{DUT {\normalfont Informatique Génie logiciel}}
{IUT de Rodez}
{}
%------------------------------------------------
\entry
{1998}
{Bac S {\normalfont Spécialité Biologie}}
{Lycée Jeanne d’Arc de Millau}
{}
%------------------------------------------------
\end{entrylist}

%----------------------------------------------------------------------------------------
%	Section COMPETENCES
%----------------------------------------------------------------------------------------

\section{compétences}

\begin{capabilitize}
	\capability {Anglais}
		{Lu, écrit et parlé}
		{Utilisé lors de mes déplacements aux USA et durant les réunions de cadrage avec les clients
		étrangers.}
    \capability {Management \& Leadership}
		{}
		{Création, structuration et management direct de 4 équipes et 20 personnes chez i-Run.fr}
    \capability {Agile, Scrum, Kanban}
		{}
		{Méthodologie utilisée avec souplesse chez Living Objects. Mise en place de rituels chez i-Run.fr}
    \capability {Pilotage de projets \& Organisation}
		{}
		{Nombreux projets pilotés ou supervisés sur l’ensemble des équipes chez i-Run.fr}
    \capability {Recrutement}
		{}
		{Chargé des recrutements informatiques chez i-Run.fr, pour faire passer l’équipe de 6 à 20 personnes}
    \capability {Pédagogie}
        {}
        {Formations à l’équipe de développement chez i-Run.fr et animations d’ateliers hebdomadaires de partage de connaissances}
    \capability {Bienveillance \& écoute}
		{}
		{Mise en place de rituels de communication entre les membres du service chez i-Run.fr pour favoriser le dialogue 
        et l’épanouissement des membres de l’équipe}
    \capability {Bagage Technique}
		{Expert}
		{Compétences Java, Spring, MySQL, Angular, Vue.js, Linux\dots\ exploités tout au long de ma carrière}
\end{capabilitize}

%----------------------------------------------------------------------------------------
%	Section INTERETS
%----------------------------------------------------------------------------------------

\section{divers \& centres d’intérêt}

Détenteur d’un \textbf{permis B}

\textbf{Professionnel} : Travail en équipe, challenges techniques, Software Craftsmanship, Brown Bag Lunch.

\textbf{Personnel} : Membre du bureau de l’association sportive VLAM (Volley-Ball).

\end{document}
