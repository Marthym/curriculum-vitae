%%%%%%%%%%%%%%%%%%%%%%%%%%%%%%%%%%%%%%%%%
% Frédéric Combes/CV
% XeLaTeX Template
% Version 1.0 (2014-12-06)
%
% This template has been downloaded from:
% http://www.LaTeXTemplates.com
%
% Original author:
% Adrien Friggeri (adrien@friggeri.net)
% https://github.com/afriggeri/CV
%
% License:
% CC BY-NC-SA 3.0 (http://creativecommons.org/licenses/by-nc-sa/3.0/)
%
% Important notes:
% This template needs to be compiled with XeLaTeX and the bibliography, if used,
% needs to be compiled with biber rather than bibtex.
%
%%%%%%%%%%%%%%%%%%%%%%%%%%%%%%%%%%%%%%%%%

\documentclass{friggeri-cv} 	% Add 'print' as an option into the square bracket to remove colors from this template for printing

\usepackage[french]{babel}
\frenchbsetup{StandardItemLabels}

\usepackage{csquotes}

\hypersetup{pdfborder={0 0 0},colorlinks}
\newcommand*{\myAddress}{
APT 6\\
282 Avenue Antoine-de-Saint-Exupéry\\
31400 TOULOUSE\\
France
}

\newcommand*{\myPhoneNumber}{+33 (0)6 31 97 71 06}

\newcommand*{\myLanguages}{
français \textit{maternelle}\\
anglais \textit{(lu, écrit, parlé)}
}

\newcommand*{\myReferences}{
\href{mailto:marthym@gmail.com}{marthym@gmail.com}
\href{https://github.com/Marthym}{github.com/Marthym}
\href{https://www.linkedin.com/in/combesfrederic}{in/combesfrederic}
}

\begin{document}

\headernamefirst{frédéric}{|}{combes}{java technical leader}

%----------------------------------------------------------------------------------------
%	SIDEBAR SECTION
%----------------------------------------------------------------------------------------

\begin{aside} % In the aside, each new line forces a line break
\section{compétences}
Java 11, Spring Boot,
Maven, Git, Docker,
MySQL, Mongo,
Agile, Scrum,
Linux
\section{contact}
\myAddress
~
\myPhoneNumber
\myReferences
\section{langues}
\myLanguages
\end{aside}

%----------------------------------------------------------------------------------------
%	SECTION: 'Expérience Professionnelle'
%----------------------------------------------------------------------------------------
\section{expériences professionnelles}

\begin{entrylist}
%---- I-RUN.FR --------------------------------------------------------------------
\entry
{Depuis 2018}
{Technical Leader Backend}
{Toulouse, Launaguet}
{\vspace{-0.4cm}\emph{i-Run.fr} \markedright{/ Spring Boot, Java 11, Vue.js, Docker, Linux, MySQL, Agile /}\\
\begin{description}[leftmargin=0cm]
    \item [\hspace*{-1cm}\bodyfont{|} \normalfont \textbf{\color{orange}CTO \color{headercolor}Adjoint}] \hfill \textit{Avril 2020}\\
    Je supervise alors l’équipe \textbf{d’administrateurs système} en plus des équipes de développement. Je participe aux développements \textbf{Java 11},valide \textbf{l’architecte logicielle} du projet de refonte et organise la mise en place du monitoring des applications grâce à \textbf{Prometheus et Grafana}. Le remplacement de machines de production par des \textbf{VM sous Proxmox} et la segmentation du \textbf{réseau en VLANs}, ont permis de sécuriser la production et de préparer l’infrastructure pour les nouvelles applications. 

    \item [\hspace*{-1cm}\bodyfont{|} \normalfont \textbf{\color{orange}Responsable \color{headercolor}Technique}] \hfill \textit{Avril 2019}\\
    J’ai initié et porté le projet de refonte et ré-architecture logicielle des applications. J’ai formé les équipes à \textbf{Java 11, Reactor et Spring Weblux} pour développer de nouveaux composants. J’ai porté les concepts d’\textbf{architecture hexagonale} auprès des équipes de développement. J’ai de plus développé des outils à base de \textbf{Docker, d’ansible ou de bash} pour faciliter les développements.

    \item[\hspace*{-1cm}\bodyfont{|} \normalfont \textbf{\color{orange}Tech \color{headercolor}Lead Backend}] \hfill \textit{Janvier 2018}\\
    Ma première tache fut de sécuriser le système de déploiement. J’ai \textbf{uniformisé et automatisé les builds} de tous les projets et développé un CI dans gitlab qui permet aujourd’hui de déployer l'ensemble des applications en un click. En parallèle, je participe aux développements et à la maintenance des applications \textbf{Java 8}. 
\end{description}
\
}

%---- LIVING OBJECTS --------------------------------------------------------------------
\entry
{2014 \ding{224} 2017}
{Expert développement Java}
{Toulouse, Basso Cambo}
{\vspace{-0.4cm}\emph{Expert développement Java} \markedright{/ Java 8, Neo4j, Maven, Docker, Scrum /}\\
\begin{enumerate}[leftmargin=-1cm]
    \item Membre d’une équipe d’une \textbf{quinzaine de développeurs}, je travaille sur le produit principal de la société.
    \item Je participe à l’élaboration de l’\textbf{architecture micro-service REST} du produit.
    \item Je développe un composant logiciel en \textbf{Java 8}, sur un \textbf{framework OSGi} pour le serveur REST. 
        Le stockage utilise une \textbf{base de données Neo4j} pour représenter les connexions entre les routeurs du réseau mobile SFR.
\end{enumerate}
\
}
\end{entrylist}
\begin{entrylist}
%---- CAMELEON SOFTWARE -----------------------------------------------------------------
\entry
{2007 \ding{224} 2013}
{Consultant Technique Expert}
{Toulouse, Labège}
{\vspace{-0.4cm}\emph{Cameleon Software} \markedright{/ Java J2EE, Struts, Jenkins, VMWare, Oracle /}\\
\begin{enumerate}[leftmargin=-1cm]
\item Intégré comme expert dans des \textbf{équipes de 10 à 20 personnes}, je développe des \textbf{composants d’interface responsive en JSP} et je forme les équipes à l’intégration de solution e-Commerce pour de grosses compagnies d’assurance comme MNT ou MMA.
\item Réalisation de PoC techniques, pour des clients comme AXA, MMA, SAGE ou encore La Poste.
\item \textbf{Missions à l’étranger} pour travailler avec notre équipe à Chicago sur des projets SaS comme Clear Channel et Crystal.
\item Mise en place et maintenance de \textbf{l’intégration continue} au sein de l’équipe technique de Caméléon (déploiement d’un serveur Git, VMWare et Jenkins).
\end{enumerate}
\
}
\end{entrylist}
\begin{entrylist}
%---- ACCESS COMMERCE --------------------------------------------------------------------
\entry
{2001 \ding{224} 2007}
{Développeur d'adaptations}
{Toulouse, Labège}
{\vspace{0.2cm}\emph{Access-Commerce}  \markedright{/ Java, J2EE, Administration Oracle, Linux/UNIX /}\\
Centre de compétences ERP chez Access-Commerce à Toulouse Labège.
En tant qu’unique ressource technique d'une équipe de sept personnes, j’ai, entre autres, effectué les
missions suivantes :
\begin{enumerate}
\item Installation et administration de serveur Oracle sur systèmes Windows et Unix
\item Développement de portail de vente en ligne pour la société Pages Jaunes
\item Développements et adaptations diverses de l’ERP OCTAL basé sur `Oracle Forms Designer`
\end{enumerate}
}
%------------------------------------------------
\end{entrylist}

%----------------------------------------------------------------------------------------
%	Section FORMATION
%----------------------------------------------------------------------------------------

\section{formation}

\begin{entrylist}
%------------------------------------------------
\entry
{2006}
{Certification Oracle {\normalfont 10g Associate}}
{Oracle, Paris}
{}
%------------------------------------------------
\entry
{2001}
{Diplôme Universitaire {\normalfont Nouvelles technologies}}
{IUT de Rodez}
{Année de spécialisation Internet et nouvelles technologies}
%------------------------------------------------
\entry
{2000}
{DUT {\normalfont Informatique Génie logiciel}}
{IUT de Rodez}
{}
%------------------------------------------------
\entry
{1998}
{Bac S {\normalfont Spécialité Biologie}}
{Lycée Jeanne d’Arc de Millau}
{}
%------------------------------------------------
\end{entrylist}

%----------------------------------------------------------------------------------------
%	Section COMPETENCES
%----------------------------------------------------------------------------------------

\section{compétences}

\begin{capabilitize}
	\capability {Anglais}
		{Lu, écrit et parlé}
		{Utilisé lors de mes déplacements aux USA et durant les réunions de cadrage avec les clients
		étrangers.}
	\capability {Java 11 / Maven}
		{Expert}
		{Langage de programmation principal des applications éditées par Caméléon Software, Living Objects \& I-Run.fr}
	\capability {Spring / Spring boot}
		{Expert}
		{Framework principal des application chez I-Run.fr}
	\capability {Angular / Vue.js}
		{Connaissance}
		{Framework front utilisé chez I-Run.fr}
	\capability {Agile, Scrum}
		{Maîtrise}
		{Méthodologie utilisée avec souplesse chez Living Objects afin de s’organiser sur le projet EYE.lo}
	\capability {Oracle, MySQL}
		{Expert}
		{Certification OCA 10g, MySQL est la base de données utilisé chez Living Objects}
	\capability {Linux}
		{Expert}
		{Unique système d’exploitation utilisé chez Cameleon Software \& Living Objects}
\end{capabilitize}

%----------------------------------------------------------------------------------------
%	Section INTERETS
%----------------------------------------------------------------------------------------

\section{divers \& centres d’intérêt}

Détenteur d’un \textbf{permis B}

\textbf{professionnel:} travail en équipe, challenges techniques, algorithmique, TDD

\textbf{personnel:} Membre du bureau de l’association sportive de Volley de Montaudran, VLAM.

\end{document}
