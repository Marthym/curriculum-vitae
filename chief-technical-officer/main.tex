%%%%%%%%%%%%%%%%%%%%%%%%%%%%%%%%%%%%%%%%%
% Frédéric Combes/CV
% XeLaTeX Template
% Version 1.0 (2014-12-06)
%
% This template has been downloaded from:
% http://www.LaTeXTemplates.com
%
% Original author:
% Adrien Friggeri (adrien@friggeri.net)
% https://github.com/afriggeri/CV
%
% License:
% CC BY-NC-SA 3.0 (http://creativecommons.org/licenses/by-nc-sa/3.0/)
%
% Important notes:
% This template needs to be compiled with XeLaTeX and the bibliography, if used,
% needs to be compiled with biber rather than bibtex.
%
%%%%%%%%%%%%%%%%%%%%%%%%%%%%%%%%%%%%%%%%%

\documentclass{friggeri-cv} 	% Add 'print' as an option into the square bracket to remove colors from this template for printing

\usepackage[french]{babel}
\frenchbsetup{StandardItemLabels}

\usepackage{csquotes}

\hypersetup{pdfborder={0 0 0},colorlinks}
\newcommand*{\myAddress}{
APT 111, BAT C\\
46 Chemin de l’église de Montaudran\\
31400 TOULOUSE\\
France
}

\newcommand*{\myPhoneNumber}{+33 (0)6 31 97 71 06}

\newcommand*{\myLanguages}{
français \textit{maternelle}\\
anglais \textit{(lu, écrit, parlé)}
}

\newcommand*{\myReferences}{
\href{mailto:marthym@gmail.com}{marthym@gmail.com}
\href{https://github.com/Marthym}{github.com/Marthym}
\href{https://www.linkedin.com/in/combesfrederic}{in/combesfrederic}
}

\begin{document}

\header{Chief Technical Officer}{frédéric}{|}{combes}

%----------------------------------------------------------------------------------------
%	SIDEBAR SECTION
%----------------------------------------------------------------------------------------

\begin{aside} % In the aside, each new line forces a line break
\section{contact}
\myAddress
~
\myPhoneNumber
\myReferences
\section{langues}
\myLanguages
\section{compétences}
Management, Leadership,
Communication,
Bienveillance,
Agile, Scrum,
Bagage Technique
\end{aside}

%----------------------------------------------------------------------------------------
%	SECTION: 'Expérience Professionnelle'
%----------------------------------------------------------------------------------------
\section{expériences professionnelles}

\begin{entrylist}
%---- i-RUN.FR --------------------------------------------------------------------
\entry
{Depuis 2018}
{CTO Adjoint}
{Toulouse, Launaguet}
{\vspace{-0.2cm}\emph{i-Run.fr} \discretright{/ Management, Organisation, Agilité, Background Technique /}\\

\begin{description}[leftmargin=0cm]
    \item [\hspace*{-1cm}\bodyfont{|} \normalfont \textbf{\color{orange}CTO \color{headercolor}Adjoint}] \hfill \textit{Avril 2020}\\
    La société me confie l’ensemble des équipes du service informatique, composé alors de \textbf{quatre équipes et de 20 personnes} réparties en développeurs, consultants fonctionnels, administrateurs système et data scientists. Ma mission première a été de structurer les équipes afin de faciliter et fluidifier les communications entre elles. En lien direct avec la direction générale de la société, j’\textbf{élabore la stratégie informatique} pour l’ensemble des sites commerciaux de l’entreprise ainsi que pour le backoffice développé en interne. Je \textbf{prends les décisions adaptées} à la stratégie de la société \textbf{en pleine autonomie et supervise les équipes} du service dans la réalisation des projets.\\
    J’ai eu à c\oe{ur} de transmettre à mes équipes des valeurs de \textbf{bienveillance et de coopération} qui ont amené le service à la \textbf{réalisation de projets majeurs} comme le déplacement de notre production de Paris à Toulouse, la restructuration de tout notre réseau informatique logistique ou l’expansion de la société en Allemagne.
    \item [\hspace*{-1cm}\bodyfont{|} \normalfont \textbf{\color{orange}Responsable \color{headercolor}Technique}] \hfill \textit{Avril 2019}\\
    La société me fait confiance et me promeut au poste de Responsable Technique. Je manage alors l’ensemble des 8 développeurs. Je \textbf{supervise l’ensemble des choix techniques}. J’organise le fonctionnent de l’équipe via des rituels agile et un processus de travail Kanban. Je fais le \textbf{lien avec les équipes} fonctionnelle et système afin de coordonner au mieux les efforts de chacun. En charge de l’équipe, \textbf{je recrute de nouveaux développeurs} nécessaires au renforcement de l’équipe.
    \item[\hspace*{-1cm}\bodyfont{|} \normalfont \textbf{\color{orange}Tech \color{headercolor}Lead Backend}] \hfill \textit{Janvier 2018} \\
    Arrivé en tant que Tech Lead Backend dans une équipe composée alors de 5 développeurs. Mes premières missions ont été, avec l’aide du Lead Frontend, de proposer \textbf{une stratégie et une architecture logicielle cohérentes}. La mise en place de \textbf{process} et de \textbf{bonnes pratiques} ont permis à l’équipe en place de réduire les régressions. L’établissement de \textbf{plans de formations} ont amené l’équipe à progresser et d’envisager plus sereinement les changements à venir.
\end{description}
\
}

%---- LIVING OBJECTS --------------------------------------------------------------------
\entry
{2014 \ding{224} 2017}
{Expert développement Java}
{Toulouse, Basso Cambo}
{\vspace{0.2cm}\emph{Living Objects} \discretright{/ Agile, Scrum, Java 8, Maven, Docker /}\\
Au sein du département Recherche \& Développement, je participe au développement du produit principal de la société.\\
L’autonomie laissée à notre équipe de quinze développeurs me permet de mettre à profit ma capacité à prendre du recul. Ma vision globale du projet m’a permis de proposer des choix techniques et des solutions appropriées aux difficultés rencontrées.
\\}
\end{entrylist}
\begin{entrylist}
%---- CAMELEON SOFTWARE -----------------------------------------------------------------
\entry
{2007 \ding{224} 2013}
{Consultant Technique Expert}
{Toulouse, Labège}
{\vspace{0.2cm}\emph{Cameleon Software} \discretright{/ Java J2EE, Struts, Jenkins /}\\
Référent Technique au sein de l’équipe "CEO Services" composée de neuf personnes, j’ai été chargé de comprendre et d’intégrer les besoins de nos clients avec la solution logicielle développée par l’entreprise. En contact direct avec les clients, les prestataires et les équipes de développement, j’ai eu la charge de coordonner les ressources techniques pour amener les projets au bout dans les délais requis.
\
}

%---- ACCESS COMMERCE --------------------------------------------------------------------
\entry
{2001 \ding{224} 2007}
{Développeur Java / Oracle}
{Toulouse, Labège}
{\vspace{0.2cm}\emph{Access-Commerce}  \discretright{/ Java, J2EE, Administration Oracle, Linux/UNIX /}\\
En tant qu’unique ressource technique d'une équipe de sept personnes, j’ai, entre autres, effectué des développements et adaptations diverses sur l’ERP intégré par la société.
}
%------------------------------------------------
\end{entrylist}

%----------------------------------------------------------------------------------------
%	Section FORMATION
%----------------------------------------------------------------------------------------

\section{formation}

\begin{entrylist}
%------------------------------------------------
\entry
{2006}
{Certification Oracle {\normalfont 10g Associate}}
{Oracle, Paris}
{}
%------------------------------------------------
\entry
{2001}
{Diplôme Universitaire {\normalfont Nouvelles technologies}}
{IUT de Rodez}
{Année de spécialisation Internet et nouvelles technologies}
%------------------------------------------------
\entry
{2000}
{DUT {\normalfont Informatique Génie logiciel}}
{IUT de Rodez}
{}
%------------------------------------------------
\entry
{1998}
{Bac S {\normalfont Spécialité Biologie}}
{Lycée Jeanne d’Arc de Millau}
{}
%------------------------------------------------
\end{entrylist}

%----------------------------------------------------------------------------------------
%	Section COMPETENCES
%----------------------------------------------------------------------------------------

\section{compétences}

\begin{capabilitize}
	\capability {Anglais}
		{Lu, écrit et parlé}
		{Utilisé lors de mes déplacements aux USA et durant les réunions de cadrage avec les clients
		étrangers.}
    \capability {Management \& Leadership}
		{}
		{Création, structuration et management direct de 4 équipes et 20 personnes chez i-Run.fr}
    \capability {Agile, Scrum, Kanban}
		{}
		{Méthodologie utilisée avec souplesse chez Living Objects. Mise en place de rituels chez i-Run.fr}
    \capability {Pilotage de projet}
		{}
		{Nombreux projets pilotés ou supervisés en tant que DSI Adjoint chez i-Run.fr}
        \capability {Organisation}{}
		{Capacité à gérer et suivre de front plusieurs projets sur plusieurs équipes chez i-Run.fr}
    \capability {Recrutement}
		{}
		{Chargé de tous les recrutements informatique chez i-Run.fr, pour faire passer l’équipe de 6 à 20 personnes}
    \capability {Bienveillance \& écoute}
		{}
		{Mise en place de rituels de communication entre les membres du service chez i-Run.fr pour favoriser le dialogue 
        et l’épanouissement des membres de l’équipe.}
    \capability {Bagage Technique}
		{Expert}
		{Compétences Java, Spring, MySQL, Angular, Linux, \dots\ exploités tout au long de ma carrière}
\end{capabilitize}

%----------------------------------------------------------------------------------------
%	Section INTERETS
%----------------------------------------------------------------------------------------

\section{divers \& centres d’intérêt}

Détenteur d’un \textbf{permis B}

\textbf{professionnel:} travail en équipe, challenges techniques, Software Craftsmanship.

\textbf{personnel:} Membre du bureau de l’association sportive VLAM (Volley-Ball).

\end{document}
