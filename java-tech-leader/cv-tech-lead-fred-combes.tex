%%%%%%%%%%%%%%%%%%%%%%%%%%%%%%%%%%%%%%%%%
% Frédéric Combes/CV
% XeLaTeX Template
% Version 1.0 (2014-12-06)
%
% This template has been downloaded from:
% http://www.LaTeXTemplates.com
%
% Original author:
% Adrien Friggeri (adrien@friggeri.net)
% https://github.com/afriggeri/CV
%
% License:
% CC BY-NC-SA 3.0 (http://creativecommons.org/licenses/by-nc-sa/3.0/)
%
% Important notes:
% This template needs to be compiled with XeLaTeX and the bibliography, if used,
% needs to be compiled with biber rather than bibtex.
%
%%%%%%%%%%%%%%%%%%%%%%%%%%%%%%%%%%%%%%%%%

\documentclass{friggeri-cv} 	% Add 'print' as an option into the square bracket to remove colors from this template for printing

\usepackage[french]{babel}
\frenchbsetup{StandardItemLabels}

\usepackage{csquotes}

\hypersetup{pdfborder={0 0 0},colorlinks}
\newcommand*{\myAddress}{
APT 111, BAT C\\
46 Chemin de l’église de Montaudran\\
31400 TOULOUSE\\
France
}

\newcommand*{\myPhoneNumber}{+33 (0)6 31 97 71 06}

\newcommand*{\myLanguages}{
français \textit{maternelle}\\
anglais \textit{(lu, écrit, parlé)}
}

\newcommand*{\myReferences}{
\href{mailto:marthym@gmail.com}{marthym@gmail.com}
\href{https://github.com/Marthym}{github.com/Marthym}
\href{https://www.linkedin.com/in/combesfrederic}{in/combesfrederic}
}

\begin{document}

\header{frédéric}{|}{combes}{java tech leader}

%----------------------------------------------------------------------------------------
%	SIDEBAR SECTION
%----------------------------------------------------------------------------------------

\begin{aside} % In the aside, each new line forces a line break
\section{contact}
\myAddress
~
\myPhoneNumber
\myReferences
\section{langues}
\myLanguages
\section{compétences}
Agile, Scrum
Java 8, Spring,
Maven, Git, Docker
Oracle, MySQL, Neo4j
Linux
\end{aside}

%----------------------------------------------------------------------------------------
%	SECTION: 'Expérience Professionnelle'
%----------------------------------------------------------------------------------------
\section{expériences professionnelles}

\begin{entrylist}
%---- I-RUN.FR --------------------------------------------------------------------
\entry
{Depuis 2018}
{I-Run.fr}
{Toulouse, Launaguet}
{\vspace{0.2cm}\emph{Technical Leader Backend} \discretright{/ Agile, Spring, Java 8, Maven, Docker /}\\
Responsable d’une équipe de 3 à 6 personnes, je suis en charge de trouver des solutions aux problématiques de développement 
et de maintenance des projets I-Run.fr.

Composées de plusieurs front office et d’un back office, les applications I-Run souffrent d’un fort legacy. 
Avec l’aide de mon homologue front, je suis en charge de repenser l’architecture des applications et d’accompagner les 
développeurs dans la transition entre l’ancienne architecture et la nouvelle.

En parallèle de ce projet de refactor, je suis en charge de maintenir la cohérence du code par des revues régulières 
et le suivi des développements et de mises en production efectuées toutes les semaines.
\\}

%---- LIVING OBJECTS --------------------------------------------------------------------
\entry
{2014 \ding{224} 2017}
{Living Objects}
{Toulouse, Basso Cambo}
{\vspace{0.2cm}\emph{Expert développement Java} \discretright{/ Agile, Scrum, Java 8, Maven, Docker /}\\
Au sein du département Recherche \& Développement de Living Objects, composé d’une quinzaine de développeurs,
réparti entre frontend et backend, je participe au développement de "eye.LO". Le produit principal de la société. Développé
en Java 8, sur un framework OSGi avec une architecture micro-service REST.\\

L’autonomie laissée à l’équipe de développement me permet de mettre à profit mon expérience et ma capacité à prendre du recul afin de proposer des choix techniques et des solutions appropriés aux difficultés rencontrées.
\\}

%---- CAMELEON SOFTWARE -----------------------------------------------------------------
\entry
{2007 \ding{224} 2013}
{Cameleon Software}
{Toulouse, Labège}
{\vspace{0.2cm}\emph{Consultant Technique Expert} \discretright{/ Java J2EE, Struts, Jenkins /}\\
Référent Technique au sein de l’équipe "CEO Services" composée de neuf personnes, mes attributions sont les suivantes :

\begin{enumerate}
\item Développement de composants d’interface sur les nouveaux produits. Pour des sociétés comme la MNT pour qui j’ai implémenté des composants de leur offre en ligne grâce à ma maîtrise des technologies web 2.0.
\item Réalisation d’intégration entre nos produits et des CRM tel que Selligent pour des clients comme Gras Savoye grâce à mes compétences en Web Services.
\item Réalisation de PoC, pour des clients comme AXA, MMA, SAGE ou encore La Poste. Mon sens du relationnel et l’esprit d’équipe nécessaires pour de telles réalisations, m’ont permis de présenter des concepts aboutis à des clients convaincus.
\item Missions à l’étranger pour travailler avec notre équipe à Chicago sur des projets SaS comme Clear Channel et Crystal Rock pour qui j'ai développé des interfaces avancées de configuration de produit.
\end{enumerate}

De plus mes initiatives ont permis de mettre en place ou d’améliorer les processus de développement et de test au sein de l’équipe technique (Serveur VMWare, intégration continue avec Jenkins, passage de SubVersion à Git, \ldots).
\
}
\end{entrylist}
\begin{entrylist}
%---- ACCESS COMMERCE --------------------------------------------------------------------
\entry
{2001 \ding{224} 2007}
{Access-Commerce}
{Toulouse, Labège}
{\vspace{0.2cm}\emph{Développeur d'adaptations}  \discretright{/ Java, J2EE, Administration Oracle, Linux/UNIX /}\\
Centre de compétences ERP chez Access-Commerce à Toulouse Labège.
En tant qu’unique ressource technique d'une équipe de sept personnes, j’ai, entre autres, effectué les
missions suivantes :
\begin{enumerate}
\item Installation et administration de serveur Oracle sur systèmes Windows et Unix
\item Développement de portail de vente en ligne pour la société Pages Jaunes
\item Développements et adaptations diverses de l’ERP OCTAL basé sur `Oracle Forms Designer`
\end{enumerate}
}
%------------------------------------------------
\end{entrylist}

%----------------------------------------------------------------------------------------
%	Section FORMATION
%----------------------------------------------------------------------------------------

\section{formation}

\begin{entrylist}
%------------------------------------------------
\entry
{2006}
{Certification Oracle {\normalfont 10g Associate}}
{Oracle, Paris}
{}
%------------------------------------------------
\entry
{2001}
{Diplôme Universitaire {\normalfont Nouvelles technologies}}
{IUT de Rodez}
{Année de spécialisation Internet et nouvelles technologies}
%------------------------------------------------
\entry
{2000}
{DUT {\normalfont Informatique Génie logiciel}}
{IUT de Rodez}
{}
%------------------------------------------------
\entry
{1998}
{Bac S {\normalfont Spécialité Biologie}}
{Lycée Jeanne d’Arc de Millau}
{}
%------------------------------------------------
\end{entrylist}

%----------------------------------------------------------------------------------------
%	Section COMPETENCES
%----------------------------------------------------------------------------------------

\section{compétences}

\begin{capabilitize}
	\capability {Anglais}
		{Lu, écrit et parlé}
		{Utilisé lors de mes déplacements aux USA et durant les réunions de cadrage avec les clients
		étrangers.}
	\capability {Java 8 / Maven}
		{Expert}
		{Langage de programmation principal des applications éditées par Caméléon Software, Living Objects \& I-Run.fr}
	\capability {Spring / Spring boot}
		{Expert}
		{Framework principal des application chez I-Run.fr}
	\capability {Angular / Vue.js}
		{Connaissance}
		{Framework front utilisé chez I-Run.fr}
	\capability {Agile, Scrum}
		{Maîtrise}
		{Méthodologie utilisée avec souplesse chez Living Objects afin de s’organiser sur le projet EYE.lo}
	\capability {Subversion, Git}
		{Maîtrise}
		{Outils de gestion de source utilisé par Cameleon Software \& Living Objects}
	\capability {Oracle, MySQL}
		{Expert}
		{Certification OCA 10g, MySQL est la base de données utilisé chez Living Objects}
	\capability {Linux}
		{Expert}
		{Unique système d’exploitation utilisé chez Cameleon Software \& Living Objects}
\end{capabilitize}

%----------------------------------------------------------------------------------------
%	Section INTERETS
%----------------------------------------------------------------------------------------

\section{divers \& centres d’intérêt}

Détenteur d’un \textbf{permis B}

\textbf{professionnel:} travail en équipe, challenges techniques, algorithmique, TDD

\textbf{personnel:} Membre du bureau de l’association sportive VLAM. Et, depuis 3 ans, capitaine de l’équipe Mixte du club.

\end{document}
